\subsection{蒸汽发生器的发展现状}
蒸汽发生器是核电动力设备中的一个主要部件,产生汽轮机所需蒸汽的换热设备\cite{1962核动力装置及其设备}。在核能反应堆中,核能产生的热量由冷却剂带出,通过蒸汽发生器传给二回路的给水,使其产生具有一定压力、一定温度和一定干度的蒸汽,此蒸汽再进入汽轮机中做功,转换为电能或机械能\cite{kim2011steam}。在这个能量转换过程中,蒸汽发生器既是一回路设备, 又是二回路设备,所以被称为一、二回路的枢纽。实际运行经验表明,蒸汽发生器能否安全、可靠地运行,对整个核动力装置的经济性和安全性具有十分重要的影响。
\par
国外压水堆核电站的运行经验表明,蒸汽发生器的性能(无论是静态性能还是动态性能)均能满足使用要求,但在可靠性方面却难以令人满意。在运行中发生蒸汽发生器传热管破损事故的装置数目,接近压水堆动力装置总数的一半。各国都把研究和改进蒸汽发生器当做完善压水堆核电技术的重要环节,并制定了庞大的研究计划,主要包括蒸汽发生器的热工水利分析;腐蚀理论和传热管材料的研制;无损探伤计数:振动、磨损、疲劳研究;改进结构设计,减少腐蚀化学物的浓缩;改进水质控制等。

\subsection{蒸汽发生器的基本技术要求}
在核动力装置中,由于一回路为带有放射性回路,而二回路为无放射性回路,因此在研制蒸汽发生器时对结构、强度、材料抗腐蚀性、密封性等都提出了很高的要求,其中最基本的技术要求为:
\begin{enumerate}
    \item 蒸汽发生器及其部件的设计,必须供给核电站在任何运行工况下所需的蒸汽量及规定的蒸汽参数。只有满足这个要求才能保证电站在任何负荷下经济运行。
    \item 蒸汽发生器的容量应该最大限度地满足功率负荷的需要,而且要求随着单机容量的增加,其技术经济指标得到相应改善。
    \item 蒸汽发生器的所有部件应该绝对地安全可靠。
    \item 蒸汽发生器个零、部件的装配必须保证在密封面上排除一回路工质漏入二回路中去的可能性。
    \item 必须排除加剧腐蚀的任何可能性,特别是一回路中的腐蚀。
    \item 蒸汽发生器必须产生必要纯度的蒸汽,以保证蒸汽发生器在高温下可靠地运行,并保证汽轮机也可靠而经济地运行。
    \item 蒸汽发生器应设计得简单紧凑,便于安装使用,同时易于发现故障而及时排除,并有可能彻底疏干。
    \item 保证蒸汽发生器具有较高的技术经济指标。
\end{enumerate}
\par
在设计蒸汽发生器时,要考虑一、二回路两种工质的种类和参数,正确地选择结构方案、材料、传热管尺寸、传热系数以及冷却剂等,对取得蒸汽发生器最佳技术经济指标是非常重要的。另外,必须采取减少向外散热损失的措施。

